\documentclass[12pt,a4paper,twoside,fleqn,notitlepage,openany]{extbook}

\usepackage[
papersize={210mm,297mm},% Specify the paper dimensions
layoutsize={210mm,297mm}, layouthoffset=0cm, layoutvoffset=0cm, % Specify the dimensions of the layout and it's horizontal and vertical distance from the beginning of the paper.
%showcrop, % Uncomment to see the layout boundaries
top=2.4cm, includehead, % Margin from top to top of the header
bottom=3.2cm, heightrounded, % Margin from Bottom
bindingoffset=10mm, % Binding margin
inner=21mm, outer=31mm, % The inner/outer edge of the layout (inner:outer ratio in two-side is 1:1.5)
marginparwidth=0cm, marginparsep=3mm, % Specify the border width and distance from the bottom of the body of the text. (These marginal notes are placed inside the outer margin of the layout.)
%columnsep=1cm % Adjust separation between columns in two column mode.
%showframe, % This option is for drawing a border around the textbody.
]{geometry}

\usepackage{hyperref}
\usepackage{fancyhdr}
\usepackage[perpage]{footmisc}
\usepackage{float}
\usepackage{adjustbox}
\usepackage{pdfpages}
\usepackage{parsa}
\usepackage[localise,fontsloadable]{xepersian}
\settextfont{IRZar}
\setlatintextfont{Linux Libertine}
\defpersianfont\ParsaHeaderFont{IRNazanin}

\fancypagestyle{plain}{\fancyhf{}\fancyfoot[OL,ER]{\thepage}\renewcommand{\headrulewidth}{0pt}}

\SepMark{-}


%----------------------------------------------------------------------------------------
%	THESIS/DISSERTATION INFORMATION STARTS FROM HERE:
%

% --- Institute's Information
\InstituteLogo{IranLogo}{IntLogo}[30mm][30mm] % {<Iranian logo>}{<International logo>}[<width>][<height>}
\Institute{(نام مؤسسه)}{(Institute's Name)} % {<persian>}{<latin>}
\Faculty{(نام دانش‌کده)}{(faculty of ...)} % {<persian>}{<latin>}
\Department{(نام گروه)}{(Department of ...)} % {<persian>}{<latin>}

% --- Student's Information
\StudentName{(نام دانشجو)}{(Student's Name)} % {<persian>}{<latin>}
\StudentMajor{(رشته)}{(Major Discipline)} % {<persian>}{<latin>}
\StudentMinor{(گرایش)}{(Minor Discipline)} % {<persian>}{<latin>}
\StudentDegree{کارشناسی‌ارشد}{(Master)} % {<persian>}{<latin>}
\StudentID{(9876543210)} % {<national ID>}
\StudentNumber{(0123456789)}
\StudentEMAIL{(sample@site.com)}

% --- Thesis/Dissertation's Information
\ParsaTitle{(عنوان): (زیرعنوان)}{(Title): (Subtitle)} % {<persian>}{<latin>}
\ParsaCompilationDate{(آذر ۱۳۹۸)}{(December, 2019)} % {<persian date>}{<latin date>}
\ParsaExamDate{(1398/09/21)}{(2019/12/12)} % {<persian date>}{<latin date>}

% --- Members of Committee's Informtion
\SupervisorName{(نام استاد راهنما)}{(Supervisor's Name)} % {<persian>}{<latin>}
\SupervisorTitle{(استاد)}{(Professor)} % {<persian>}{<latin>}
\SupervisorAffiliation{(نام مؤسسه)}{(Institute's Name)} % {<persian>}{<latin>}
\SupervisorID{(9876543210)} %{<national ID>}
\SupervisorEMAIL{(example@institute.edu)}

\SecondSupervisorName{(نام استاد راهنمای دوم)}{(Second Supervisor's Name)} % {<persian>}{<latin>}
\SecondSupervisorTitle{(دانشیار)}{(Associated Professor)} % {<persian>}{<latin>}
%\SecondSupervisorAffiliation{}{} % {<persian>}{<latin>}
%\SecondSupervisorID{} % {<national ID>}
%\SecondSupervisorEMAIL{}

\CosupervisorName{(نام استاد مشاور اول)}{(First Co-superviser's Name)} % {<persian>}{<latin>}
\CosupervisorTitle{(استادیار)}{(Assistant Prosessor)} % {<persian>}{<latin>}
%\CosupervisorAffiliation{}{} % {<persian>}{<latin>}
%\CosupervisorID{} % {<national ID>}
%\CosupervisorEMAIL{}

\SecondCosupervisorName{(نام استاد مشاور دوم)}{(Second Co-supervisor's Name)} % {<persian>}{<latin>}
%\SecondCosupervisorTitle{}{} % {<persian>}{<latin>}
%\SecondCosupervisorAffiliation{}{} % {<persian>}{<latin>}
%\SecondCosupervisorID{} % {<national ID>}
%\SecondCosupervisorEMAIL{}

\FirstExaminerName{(نام استاد داور)}{~} % {<persian>}{<latin>}
%\FirstExaminerTitle{}{} % {<persian>}{<latin>}
%\FirstExaminerAffiliation{}{} % {<persian>}{<latin>}
%\FirstExaminerID{} % {<national ID>}
%\FirstExaminerEMAIL{}

\SecondExaminerName{(نام استاد داور)}{~} % {<persian>}{<latin>}
%\SecondExaminerTitle{}{} % {<persian>}{<latin>}
%\SecondExaminerAffiliation{}{} % {<persian>}{<latin>}
%\SecondExaminerID{} % {<national ID>}
%\SecondExaminerEMAIL{}

\ThirdExaminerName{(نام استاد داور)}{} % {<persian>}{<latin>}
%\ThirdExaminerTitle{}{} % {<persian>}{<latin>}
%\ThirdExaminerAffiliation{}{} % {<persian>}{<latin>}
%\ThirdExaminerID{} % {<national ID>}
%\ThirdExaminerEMAIL{}

\RepresentativeName{(نام نماینده تحصیلات تکمیلی)}{(The representative's name)} % {<persian>}{<latin>}
%\RepresentativeID{} % {<national ID>}
%\RepresentativeEMAIL{}

%
%	THESIS/DISSERTATION INFORMATION ENDED.
%----------------------------------------------------------------------------------------



\begin{document}
\pagestyle{plain}
\عنوان{بسته‌ی پارسا}
\نویسنده{فرشاد رسولی \thanks{\href{https://github.com/farshadrasuli/parsa}{\lr{github.com/farshadrasuli/parsa}}}}
\تاریخ{1398/09/21 - \lr{December 12, 2019}}
\عنوان‌ساز
\begin{center}
\begin{minipage}{0.67\textwidth}
تألیف پایان‌نامه و رساله، چهارچوب از پیش تعیین شده‌ای مانند چگونگی درج سربرگ، ساخت و تکمیل فرم‌های مربوطه و تدوین عناصر مختلف دارد\@. این بسته با هدف تولید فرم‌های مورد نیاز، به منطور تسریع فرآیند تألیف پایان‌نامه و رساله، توسط دانشجو-پژوهش‌گر تهیه شده‌است\@. \\
امیدوارم با استفاده از این بسته، در مدت زمان تهیه‌ی پایان‌نامه و رساله، صرفه‌جویی کافی حاصل شود و به جای صرف وقت به ویرایش پایان‌نامه، زمان بیشتری به مسائل پژوهشی اختصاص یابد\@.
\end{minipage}
\end{center}
%\tableofcontents
\section*{سپاس‌گزاری}
\addcontentsline{toc}{section}{سپاس‌گزاری}
از جناب آقای \موکد{وفا کارن‌پهلو} بابت تهیه و توسعه‌ی بسته‌ی \موکد{زی‌پرشین} سپاس‌گزاری می‌کنم و برای ایشان آرزوی سلامتی و سعادت دارم\@. کار بی‌نظیر ایشان در تهیه‌ی بسته‌ی زی‌پرشین، الهام بخش تهیه‌ی بسته‌ی \موکد{پارسا} بوده‌است.

\section{شروع کار}
با استفاده از فرمان
\[\textrm{\lr{\texttt{$\backslash$usepackage[\textrm{<options>}]\{parsa\}}}}\]
بسته‌ی پارسا فعال می‌شود\@. برای گزینه‌های این بسته، \متن‌لاتین{[\emph{<options>}]}، قسمت (\رجوع{ParsaInf}) را ببینید.\\
سپس بسته‌ی \XePersian ~را فعال کنید\@.
\[ \textrm{ \lr{ \texttt{$\backslash$usepackage\{xepersian\}} } } \]
قلم اصلی متن را با فرمان
\[ \textrm{ \lr{ \texttt{$\backslash$settextfont\{\textrm{<font>}\} } } } \]
و قلم سربرگ را با فرمان
\[ \textrm{ \lr{ \texttt{$\backslash$defpersianfont$\backslash$ParsaHeaderFont\{\textrm{<font>}\} } } } \]
تعیین کنید\@. توصیه می‌شود برای قلم اصلی، از قلم زر \زیرنویس{توسعه‌دهندگان مختلف، نسخه‌های متفاوتی از قلم زر با نام‌های \متن‌لاتین{B Zar}،\متن‌لاتین{XB Zar} و \متن‌لاتین{IRZar} ~منتشر کرده‌اند که استفاده از \متن‌لاتین{IRZar} ~توصیه می‌شود.} و برای قلم سربرگ، از قلم نازنین \زیرنویس{از بین نسخه‌های متفاوت قلم نازنین با نام‌های \متن‌لاتین{B Nazanin}، \متن‌لاتین{XB Nazanin} و \متن‌لاتین{IRNazanin} ~استفاده از \متن‌لاتین{IRNazanin} ~توصیه می‌شود.} استفاده کنید\@.

\section{ورود اطلاعات}
برای وارد کردن اطلاعات پایان‌نامه (یا رساله)، فرمان مد نظر خود را، مطابق دستورات شرح داده‌شده، در قسمت سرآغاز \زیرنویس{\متن‌لاتین{preamble}}~وارد کنید.
\زیرقسمت{اطلاعات مؤسسه}
نام مؤسسه‌ی محل تحصیل خود را با فرمان
\[ \textrm{ \lr{ \texttt{$\backslash$Institute\{\textrm{<persian>}\}\{\textrm{<latin>}\}} } } \]
وارد کنید. دقت شود که آرگومان اول، \{\emph{<persian>}\}، برای ورود اطلاعات به زبان پارسی و آرگومان دوم، \{\emph{<persian>}\}، برای ورود اطلاعات به لاتین درنظر گرفته شده است. این اصل برای تمام فرمان‌های این بسته که دو آرگومان دارند رعایت شده است\@. به همین صورت، برای سایر اطلاعات از فرمان‌ها به شرح زیر استفاده کنید\@. \\
نشان مؤسسه:
\[ \textrm{ \lr{ \texttt{$\backslash$InstituteLogo\{\textrm{<Iranian logo>}\}\{\textrm{<International logo>}\}[\textrm{<width>}][\textrm{<height>}]} } } \]
با آرگومان‌های اختیاری، پهنا و بلندای نشان مؤسسه را در صفحه‌ی عنوان تنظیم کنید. مقدار پیش‌فرض ۳۰×۳۰ میلی‌متر است\@. \\
نام دانش‌کده یا پژوهش‌کده:
\[ \textrm{ \lr{ \texttt{$\backslash$Faculty\{\textrm{<persian>}\}\{\textrm{<latin>}\}} } } \]
نام گروه:
\[ \textrm{ \lr{ \texttt{$\backslash$Department\{\textrm{<persian>}\}\{\textrm{<latin>}\}} } } \]
\subsection{اطلاعات نگارنده}
برای وارد کردن اطلاعات نگارنده، از فرمان‌های زیر استفاده کنید\@. چنان‌چه قصد دارید که یکی از دو آرگومان فرمانی را خالی بگذارید، کافی‌است از نشانه‌ی مد ($\sim$) استفاده کنید\@. \\
نام کامل نگارنده‌ی اثر:
\[ \textrm{ \lr{ \texttt{$\backslash$StudentName\{\textrm{<persian>}\}\{\textrm{<latin>}\}} } } \]
رشته‌ی تحصیلی:
\[ \textrm{ \lr{ \texttt{$\backslash$StudentMajor\{\textrm{<persian>}\}\{\textrm{<latin>}\}} } } \]
گرایش تحصیلی:
\[ \textrm{ \lr{ \texttt{$\backslash$StudentMinor\{\textrm{<persian>}\}\{\textrm{<latin>}\}} } } \]
مقطع تحصیلی:
\[ \textrm{ \lr{ \texttt{$\backslash$StudentDegree\{\textrm{<persian>}\}\{\textrm{<latin>}\}} } } \]
شماره ملی:
\[ \textrm{ \lr{ \texttt{$\backslash$StudentID\{\textrm{<national ID number>}\}} } } \]
شماره دانشجویی:
\[ \textrm{ \lr{ \texttt{$\backslash$StudentNumber\{\textrm{<Student Number>}\}} } } \]
رایانامه:
\[ \textrm{ \lr{ \texttt{$\backslash$StudentEMAIL\{\textrm{<sample@site.com>}\}} } } \]
\subsection{اطلاعات سند} \label{ParsaInf}
باید نوع سندی که قصد تهیه‌‌ی آن را دارید، از آن جهت که پایان‌نامه است یا رساله، مشخص کنید\@. هنگام فراخوانی بسته‌ی پارسا، امکان استفاده از گزینه‌ی \متن‌لاتین{\متن‌تایپ{thesis}} ~یا \متن‌لاتین{\متن‌تایپ{dissertation}} ~وجود دارد که گزینه‌ی \متن‌لاتین{\متن‌تایپ{thesis}} ~نوع سند را پایان‌نامه و گزینه‌ی \متن‌لاتین{\متن‌تایپ{dissertation}} ~نوع سند را رساله تعیین می‌کند. چنان‌چه از این گزینه‌ها استفاده نکردید، می‌توانید با فرمان
\[ \textrm{ \lr{ \texttt{$\backslash$ParsaTarget\{\textrm{<persian>}\}\{\textrm{<latin>}\}} } } \]
نوع سند خود را مشخص کنید\@. برای سایر اطلاعات سند، از فرمان‌های زیر استفاده کنید\@. \\
عنوان سند:
\[ \textrm{ \lr{ \texttt{$\backslash$ParsaTitle\{\textrm{<persian>}\}\{\textrm{<latin>}\}} } } \]
تاریخ تدوین که در صفحه‌ی عنوان درج می‌شود (به صورت ماه و سال وارد شود):
\[ \textrm{ \lr{ \texttt{$\backslash$ParsaCompilationDate\{\textrm{<persian date>}\}\{\textrm{<International date>}\}} } } \]
تاریخ دقیق دفاع از پایان‌نامه یا رساله:
\[ \textrm{ \lr{ \texttt{$\backslash$ParsaExamDate\{\textrm{<persian date>}\}\{\textrm{<International date>}\}} } } \]
\subsection{اطلاعات استاد راهنما} \label{SupInf}
اطلاعات استاد راهنما را با فرمان‌های زیر وارد کنید\@. \\
نام کامل استاد راهنما:
\[ \textrm{ \lr{ \texttt{$\backslash$SupervisorName\{\textrm{<persian>}\}\{\textrm{<latin>}\}} } } \]
سمت یا مرتبه‌ی علمی:
\[ \textrm{ \lr{ \texttt{$\backslash$SupervisorTitle\{\textrm{<persian>}\}\{\textrm{<latin>}\}} } } \]
وابستگی سازمانی:
\[ \textrm{ \lr{ \texttt{$\backslash$SupervisorAffiliation\{\textrm{<persian>}\}\{\textrm{<latin>}\}} } } \]
شماره ملی:
\[ \textrm{ \lr{ \texttt{$\backslash$SupervisorID\{\textrm{<national ID number>}\}} } } \]
رایانامه:
\[ \textrm{ \lr{ \texttt{$\backslash$SupervisorEMAIL\{\textrm{<sample@site.com>}\}} } } \]
\subsection{اطلاعات استاد راهنمای دوم}
مشابه استاد راهنما، قسمت (\رجوع{SupInf})، اطلاعات استاد راهنمای دوم -در صورت وجود- را با فرمان‌های زیر وارد کنید\@. \\
نام کامل استاد راهنمای دوم:
\[ \textrm{ \lr{ \texttt{$\backslash$SecondSupervisorName\{\textrm{<persian>}\}\{\textrm{<latin>}\}} } } \]
سمت یا مرتبه‌ی علمی:
\[ \textrm{ \lr{ \texttt{$\backslash$SecondSupervisorTitle\{\textrm{<persian>}\}\{\textrm{<latin>}\}} } } \]
وابستگی سازمانی:
\[ \textrm{ \lr{ \texttt{$\backslash$SecondSupervisorAffiliation\{\textrm{<persian>}\}\{\textrm{<latin>}\}} } } \]
شماره ملی:
\[ \textrm{ \lr{ \texttt{$\backslash$SecondSupervisorID\{\textrm{<national ID number>}\}} } } \]
رایانامه:
\[ \textrm{ \lr{ \texttt{$\backslash$SecondSupervisorEMAIL\{\textrm{<sample@site.com>}\}} } } \]
\subsection{اطلاعات استاد مشاور}
مشابه استاد راهنما، قسمت (\رجوع{SupInf})، اطلاعات استاد مشاور -در صورت وجود- را با فرمان‌های زیر وارد کنید\@. \\
نام کامل استاد مشاور:
\[ \textrm{ \lr{ \texttt{$\backslash$CosupervisorName\{\textrm{<persian>}\}\{\textrm{<latin>}\}} } } \]
سمت یا مرتبه‌ی علمی:
\[ \textrm{ \lr{ \texttt{$\backslash$CosupervisorTitle\{\textrm{<persian>}\}\{\textrm{<latin>}\}} } } \]
وابستگی سازمانی:
\[ \textrm{ \lr{ \texttt{$\backslash$CosupervisorAffiliation\{\textrm{<persian>}\}\{\textrm{<latin>}\}} } } \]
شماره ملی:
\[ \textrm{ \lr{ \texttt{$\backslash$CosupervisorID\{\textrm{<national ID number>}\}} } } \]
رایانامه:
\[ \textrm{ \lr{ \texttt{$\backslash$CosupervisorEMAIL\{\textrm{<sample@site.com>}\}} } } \]
\subsection{اطلاعات استاد مشاور دوم}
مشابه استاد راهنما، قسمت (\رجوع{SupInf})، اطلاعات استاد مشاور دوم -در صورت وجود- را با فرمان‌های زیر وارد کنید\@. \\
نام کامل استاد مشاور دوم:
\[ \textrm{ \lr{ \texttt{$\backslash$SecondCosupervisorName\{\textrm{<persian>}\}\{\textrm{<latin>}\}} } } \]
سمت یا مرتبه‌ی علمی:
\[ \textrm{ \lr{ \texttt{$\backslash$SecondCosupervisorTitle\{\textrm{<persian>}\}\{\textrm{<latin>}\}} } } \]
وابستگی سازمانی:
\[ \textrm{ \lr{ \texttt{$\backslash$SecondCosupervisorAffiliation\{\textrm{<persian>}\}\{\textrm{<latin>}\}} } } \]
شماره ملی:
\[ \textrm{ \lr{ \texttt{$\backslash$SecondCosupervisorID\{\textrm{<national ID number>}\}} } } \]
رایانامه:
\[ \textrm{ \lr{ \texttt{$\backslash$SecondCosupervisorEMAIL\{\textrm{<sample@site.com>}\}} } } \]
\subsection{اطلاعات استاد داور اول}
مشابه استاد راهنما، قسمت (\رجوع{SupInf})، اطلاعات استاد داور اول را با فرمان‌های زیر وارد کنید\@. \\
نام کامل استاد داور اول:
\[ \textrm{ \lr{ \texttt{$\backslash$FirstExaminerName\{\textrm{<persian>}\}\{\textrm{<latin>}\}} } } \]
سمت یا مرتبه‌ی علمی:
\[ \textrm{ \lr{ \texttt{$\backslash$FirstExaminerTitle\{\textrm{<persian>}\}\{\textrm{<latin>}\}} } } \]
وابستگی سازمانی:
\[ \textrm{ \lr{ \texttt{$\backslash$FirstExaminerAffiliation\{\textrm{<persian>}\}\{\textrm{<latin>}\}} } } \]
شماره ملی:
\[ \textrm{ \lr{ \texttt{$\backslash$FirstExaminerID\{\textrm{<national ID number>}\}} } } \]
رایانامه:
\[ \textrm{ \lr{ \texttt{$\backslash$FirstExaminerEMAIL\{\textrm{<sample@site.com>}\}} } } \]
\subsection{اطلاعات استاد داور دوم}
مشابه استاد راهنما، قسمت (\رجوع{SupInf})، اطلاعات استاد داور دوم -در صورت وجود- را با فرمان‌های زیر وارد کنید\@. \\
نام کامل استاد داور دوم:
\[ \textrm{ \lr{ \texttt{$\backslash$SecondExaminerName\{\textrm{<persian>}\}\{\textrm{<latin>}\}} } } \]
سمت یا مرتبه‌ی علمی:
\[ \textrm{ \lr{ \texttt{$\backslash$SecondExaminerTitle\{\textrm{<persian>}\}\{\textrm{<latin>}\}} } } \]
وابستگی سازمانی:
\[ \textrm{ \lr{ \texttt{$\backslash$SecondExaminerAffiliation\{\textrm{<persian>}\}\{\textrm{<latin>}\}} } } \]
شماره ملی:
\[ \textrm{ \lr{ \texttt{$\backslash$SecondExaminerID\{\textrm{<national ID number>}\}} } } \]
رایانامه:
\[ \textrm{ \lr{ \texttt{$\backslash$SecondExaminerEMAIL\{\textrm{<sample@site.com>}\}} } } \]
\subsection{اطلاعات استاد داور سوم}
مشابه استاد راهنما، قسمت (\رجوع{SupInf})، اطلاعات استاد داور سوم -در صورت وجود- را با فرمان‌های زیر وارد کنید\@. \\
نام کامل استاد داور سوم:
\[ \textrm{ \lr{ \texttt{$\backslash$ThirdExaminerName\{\textrm{<persian>}\}\{\textrm{<latin>}\}} } } \]
سمت یا مرتبه‌ی علمی:
\[ \textrm{ \lr{ \texttt{$\backslash$ThirdExaminerTitle\{\textrm{<persian>}\}\{\textrm{<latin>}\}} } } \]
وابستگی سازمانی:
\[ \textrm{ \lr{ \texttt{$\backslash$ThirdExaminerAffiliation\{\textrm{<persian>}\}\{\textrm{<latin>}\}} } } \]
شماره ملی:
\[ \textrm{ \lr{ \texttt{$\backslash$ThirdExaminerID\{\textrm{<national ID number>}\}} } } \]
رایانامه:
\[ \textrm{ \lr{ \texttt{$\backslash$ThirdExaminerEMAIL\{\textrm{<sample@site.com>}\}} } } \]
\subsection{اطلاعات نماینده‌ی تحصیلات تکمیلی}
مشابه استاد راهنما، قسمت (\رجوع{SupInf})، اطلاعات نماینده‌ی تحصیلات تکمیلی را با فرمان‌های زیر وارد کنید\@. \\
نام کامل نماینده‌ی تحصیلات تکمیلی:
\[ \textrm{ \lr{ \texttt{$\backslash$RepresentativeName\{\textrm{<persian>}\}\{\textrm{<latin>}\}} } } \]
شماره ملی:
\[ \textrm{ \lr{ \texttt{$\backslash$RepresentativeID\{\textrm{<national ID number>}\}} } } \]
رایانامه:
\[ \textrm{ \lr{ \texttt{$\backslash$RepresentativeEMAIL\{\textrm{<sample@site.com>}\}} } } \]
\section{ساخت فرم‌ها}
پس از تکمیل اطلاعت لازم، با استفاده از فرمان‌های گفته‌شده، کار اصلی بسته‌ی پارسا آغاز می‌گردد\@. برای ساخت فرم‌های مدنظر، از دستورات زیر استفاده کنید.\\

\begin{table}[H] \adjustbox{pagecenter}{
\begin{tabular}{lr}
\lr{\texttt{$\backslash$titlepageParsi}}                    & صفحه‌ی عنوان به فارسی       \\
\lr{\texttt{$\backslash$ParsaPicture}}                      & عکس سرآغاز                  \\
\lr{\texttt{$\backslash$ParsaCredit}}                       & اصالت و مالکیت اثر          \\
\lr{\texttt{$\backslash$ExaminationReportPa}}               & صورت جلسه‌ی دفاع به فارسی   \\
\lr{\texttt{$\backslash$ParsaCopyleft}}                     & مجوز بهره‌برداری            \\
\lr{\texttt{$\backslash$ParsaDedicate\{\rl{تقدیم به...}\}}} & تقدیم                       \\
\lr{\texttt{$\backslash$ExaminationReportLa}}               & صورت جلسه‌ی دفاع به انگلیسی \\
\lr{\texttt{$\backslash$titlepageLatin}}                    & صفحه عنوان به انگلیسی      
\end{tabular} }
\end{table}

توجه کنید که فقط فرمان تقدیم، \textrm{\lr{\texttt{$\backslash$ParsaDedicate}}}~، به آرگومان نیاز دارد\@. متن تقدیم خود را در آرگومان این فرمان بنویسید\@. \بند
\پاراگراف{}برای نوشتن چکیده‌ی پایان‌نامه یا رساله، به زبان فارسی از محیط \متن‌لاتین{\متن‌تایپ{ParsaAbstractParsi}}~، و برای چکیده به لاتین از محیط \متن‌لاتین{\متن‌تایپ{ParsaAbstractLatin}} استفاده کنید\@. \\
\begin{latin}
\begin{minipage}{0.48\textwidth}
\begin{verbatim}
\begin{ParsaAbstractParsi}

\end{ParsaAbstractParsi}
\end{verbatim}
\end{minipage}
\begin{minipage}{0.48\textwidth}
\begin{verbatim}
\begin{ParsaAbstractParsi}

\end{ParsaAbstractParsi}
\end{verbatim}
\end{minipage}
\end{latin}

\section{کدهای آماده}
کدهای زیر برای استفاده در \LaTeX ~برای شما آماده شده‌است. برای استفاده از هر کد، کافی‌است علامت درصد ابتدای هرخط را حذف کنید.
\begin{latin}
\begin{verbatim}
\documentclass[a4paper,12pt,twoside]{book}

\usepackage{parsa}
\usepackage[fontsloadable]{xepersian}
\settextfont{} 
\defpersianfont\ParsaHeaderFont{} 

%
%	THESIS/DISSERTATION INFORMATION STARTS FROM HERE:
%

% --- Institute's Information
%\InstituteLogo{}{}[][] 
%\Institute{}{}
%\Faculty{}{}
%\Department{}{} 

% --- Student's Information
%\StudentName{}{} 
%\StudentMajor{}{}
%\StudentMinor{}{}
%\StudentDegree{}{}
%\StudentID{}
%\StudentNumber{}
%StudentEMAIL{}

% --- Thesis/Dissertation's Information
%\ParsaTitle{}{} % {<persian>}{<latin>}
%\ParsaCompilationDate{}{} % {<persian date>}{<latin date>}
%\ParsaExamDate{}{} % {<persian date>}{<latin date>}

% --- Members of The Committee's Informtion
%\SupervisorName{}{} 
%\SupervisorTitle{}{}
%\SupervisorAffiliation{}{}
%\SupervisorID{}
%\SupervisorEMAIL{}

%\SecondSupervisorName{}{}
%\SecondSupervisorTitle{}{}
%\SecondSupervisorAffiliation{}{}
%\SecondSupervisorID{}
%\SecondSupervisorEMAIL{}

%\CosupervisorName{}{}
%\CosupervisorTitle{}{}
%\CosupervisorAffiliation{}{}
%\CosupervisorID{}
%\CosupervisorEMAIL{}

%\SecondCosupervisorName{}{}
%\SecondCosupervisorTitle{}{}
%\SecondCosupervisorAffiliation{}{} %
%\SecondCosupervisorID{}
%\SecondCosupervisorEMAIL{}

%\FirstExaminerName{}{}
%\FirstExaminerTitle{}{}
%\FirstExaminerAffiliation{}{}
%\FirstExaminerID{}
%\FirstExaminerEMAIL{}

%\SecondExaminerName{}{}
%\SecondExaminerTitle{}{}
%\SecondExaminerAffiliation{}{}
%\SecondExaminerID{}
%\SecondExaminerEMAIL{}

%\ThirdExaminerName{}{}
%\ThirdExaminerTitle{}{}
%\ThirdExaminerAffiliation{}{}
%\ThirdExaminerID{}
%\ThirdExaminerEMAIL{}

%\RepresentativeName{}{}
%\RepresentativeID{}
%\RepresentativeEMAIL{}


\begin{document}

\titlepageParsi
\newpage

%-----------------------------------------------
%	PERSIAN ABSTRACT STARTS FROME HERE:
%
\begin{ParsaAbstractParsi}

\end{ParsaAbstractParsi}
%
%	PERSIAN ABSTRACT ENDED.
%-----------------------------------------------

\ParsaPicture{}

\ParsaCredit
\newpage

\ExaminationReportPa
\newpage

\ParsaCopyleft
\newpage

\ParsaDedicate{}

\begin{ParsaAcknowledge}

\end{ParsaAcknowledge}
\newpage

\begin{ParsaCV}

\end{ParsaCV}

\ExaminationReportLa
\newpage

%-----------------------------------------------
%	LATIN ABSTRACT STARTS FROME HERE:
%
\begin{ParsaAbstractLatin}

\end{ParsaAbstractLatin}
%
%	LATIN ABSTRACT ENDED.
%-----------------------------------------------

\titlepageLatin

\end{document}
\end{verbatim}
\end{latin}

\section{نمونه پایان‌نامه}
در صفحات بعدی، نمونه‌ای از پایان نامه‌ی انجام‌شده به کمک بسته‌ی پارسا قرار گرفته‌است. \\
\newpage

\titlepageParsi
\newpage
%----------------------------------------------------------------------------------------
%	PERSIAN ABSTRACT STARTS FROME HERE:
%
\begin{ParsaAbstractParsi}
چکیده شامل خلاصه‌ای از هدف یا مسأله‌ی پژوهش، روش‌شناسی، نتایج و تفسیر می‌باشد که خواننده با مطالعه‌ی آن از محتوای پژوهش آگاه می‌شود. در چکیده از اشاره به تاریخچه، تفصیل اقوال، توصیف تکنیک‌ها، فصل بندی، ذکر منابع و آوردن فرمول‌ها، نمودارها و جداول پرهیز شود. متن چکیده حداکثر دارای سی‌صد کلمه باشد و در یک صفحه و در یک پاراگراف نوشته شود. همچنین واژگان کلیدی در یک سطر جداگانه درج می‌شود و تعداد آن بین پنح تا هشت کلمه است. \بند \bigskip

\متن‌سیاه{کلمات کلیدی:} پایان‌نامه، رساله، شیوه‌نامه، قالب آماده، زی‌پرشین، پارسا
\end{ParsaAbstractParsi}
%
%	PERSIAN ABSTRACT ENDED.
%----------------------------------------------------------------------------------------
\ParsaPicture{picture}

\ParsaCredit
\newpage

\ExaminationReportPa
\newpage

\ParsaCopyleft
\newpage

\ParsaDedicate{تقدیم به ...}

\begin{ParsaAcknowledge}
متن سپاس‌گذاری...
\end{ParsaAcknowledge}
\newpage


\chapter{فصل نمونه}
محتوای اصلی نوشتار از اینجا آغاز می‌شود...

\pagestyle{empty}

\begin{ParsaCV}
کارنامک، شمایی کوتاه از کارهای علمی و درجه‌های تحصیلی دانش‌آموخته را نشان می‌دهد و بهتر است به زبان سوم شخص (غایب) نوشته شود. این بخش برای دانش‌آموختگان کارشناسی ارشد، اختیاری و برای دانش‌آموختگان دکترا الزامی است. برای کسانی که پارسا را به زبانی به‌جز فارسی می‌نویسند این کارنامک نیز باید به همان زبان نوشته شود. نمونه‌ای از این کارنامک در ادامه نوشته شده است. \\

رضا تهرانی دانش‌آموخته‌ی دکترای تخصصی رشته‌ی زبان و ادبیات فارسی از دانشگاه ایران در گرایش نگارش علمی در سال ۱۳۹۶ است. او در سال ۱۳۹۰ کارشناسی ارشد خود را از دانشگاه ایران در رشته‌ی تاریخ ادبیات گرایش نگارش در نوشتار علمی و کارشناسی خود را در سال ۱۳۸۷ از دانشگاه ایران در رشته‌ی تاریخ ادبیات دریافت کرد. زمینه‌های پژوهش او نوشتارهای علمی، تاریخ ادبیات و ویرایش تخصصی است. \\
\end{ParsaCV}

\ExaminationReportLa
\newpage

\titlepageLatin

\end{document}
